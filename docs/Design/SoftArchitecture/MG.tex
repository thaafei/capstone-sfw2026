\documentclass[12pt, titlepage]{article}

\usepackage{fullpage}
\usepackage[round]{natbib}
\usepackage{multirow}
\usepackage{booktabs}
\usepackage{tabularx}
\usepackage{graphicx}
\usepackage{float}
\usepackage{hyperref}
\hypersetup{
    colorlinks,
    citecolor=blue,
    filecolor=black,
    linkcolor=red,
    urlcolor=blue
}

\input{../../Comments}
%% Common Parts

\newcommand{\progname}{SFWRENG 4G06 - Capstone Design Process}
\newcommand{\authname}{\textbf{Team 17, DomainX} \\
\\ Awurama Nyarko
\\ Haniye Hamidizadeh
\\ Fei Xie
\\ Ghena Hatoum             
}
\newcommand{\projname}{DomainX}
\usepackage{hyperref}
    \hypersetup{colorlinks=true, linkcolor=blue, citecolor=blue, filecolor=blue,
                urlcolor=blue, unicode=false}
    \urlstyle{same}
                                


\newcounter{acnum}
\newcommand{\actheacnum}{AC\theacnum}
\newcommand{\acref}[1]{AC\ref{#1}}

\newcounter{ucnum}
\newcommand{\uctheucnum}{UC\theucnum}
\newcommand{\uref}[1]{UC\ref{#1}}

\newcounter{mnum}
\newcommand{\mthemnum}{M\themnum}
\newcommand{\mref}[1]{M\ref{#1}}

\begin{document}

\title{Module Guide for \progname{}} 
\author{\authname}
\date{\today}

\maketitle

\pagenumbering{roman}

\section{Revision History}

\begin{table}[hp]
\caption{Revision History} \label{MgRevHis}
\begin{tabularx}{\textwidth}{llX}
\toprule
\textbf{Date} & \textbf{Developer(s)} & \textbf{Change}\\
\midrule
Nov 3, 2025 & Fei & Rev -1\\
\bottomrule
\end{tabularx}
\end{table}

\newpage

\section{Reference Material}

This section records information for easy reference.

\subsection{Abbreviations and Acronyms}

\renewcommand{\arraystretch}{1.2}
\begin{tabular}{l l} 
  \toprule		
  \textbf{symbol} & \textbf{description}\\
  \midrule 
  AC & Anticipated Change\\
  DAG & Directed Acyclic Graph \\
  M & Module \\
  MG & Module Guide \\
  OS & Operating System \\
  R & Requirement\\
  SC & Scientific Computing \\
  SRS & Software Requirements Specification\\
  \progname & Software Engineering Capstone Project\\
  UC & Unlikely Change \\
  AHP & Analytical Hierarchy Process \\ 
  BWM & Best-Worst Method \\
  SSB & Skew-Symmetric Bilinear \\ 
  Domain & Research Software Domain \\
  Packages & Software Packages \\  
  API & Application Programming Interface \\
  ADT & Abstract Data Type\\
  \bottomrule
\end{tabular}\\

\newpage

\tableofcontents

\listoftables

\listoffigures

\newpage

\pagenumbering{arabic}

\section{Introduction}

Decomposing a system into modules is a commonly accepted approach to developing
software.  A module is a work assignment for a programmer or programming
team~\citep{ParnasEtAl1984}.  We advocate a decomposition
based on the principle of information hiding~\citep{Parnas1972a}.  This
principle supports design for change, because the ``secrets'' that each module
hides represent likely future changes.  Design for change is valuable in SC,
where modifications are frequent, especially during initial development as the
solution space is explored.  

Our design follows the rules layed out by \citet{ParnasEtAl1984}, as follows:
\begin{itemize}
\item System details that are likely to change independently should be the
  secrets of separate modules.
\item Each data structure is implemented in only one module.
\item Any other program that requires information stored in a module's data
  structures must obtain it by calling access programs belonging to that module.
\end{itemize}

After completing the first stage of the design, the Software Requirements
Specification (SRS), the Module Guide (MG) is developed~\citep{ParnasEtAl1984}. The MG
specifies the modular structure of the system and is intended to allow both
designers and maintainers to easily identify the parts of the software.  The
potential readers of this document are as follows:

\begin{itemize}
\item New project members: This document can be a guide for a new project member
  to easily understand the overall structure and quickly find the
  relevant modules they are searching for.
\item Maintainers: The hierarchical structure of the module guide improves the
  maintainers' understanding when they need to make changes to the system. It is
  important for a maintainer to update the relevant sections of the document
  after changes have been made.
\item Designers: Once the module guide has been written, it can be used to
  check for consistency, feasibility, and flexibility. Designers can verify the
  system in various ways, such as consistency among modules, feasibility of the
  decomposition, and flexibility of the design.
\end{itemize}

The rest of the document is organized as follows. Section
\ref{SecChange} lists the anticipated and unlikely changes of the software
requirements. Section \ref{SecMH} summarizes the module decomposition that
was constructed according to the likely changes. Section \ref{SecConnection}
specifies the connections between the software requirements and the
modules. Section \ref{SecMD} gives a detailed description of the
modules. Section \ref{SecTM} includes two traceability matrices. One checks
the completeness of the design against the requirements provided in the SRS. The
other shows the relation between anticipated changes and the modules. Section
\ref{SecUse} describes the use relation between modules.

\section{Anticipated and Unlikely Changes} \label{SecChange}

This section lists possible changes to the system. According to the likeliness
of the change, the possible changes are classified into two
categories. Anticipated changes are listed in Section \ref{SecAchange}, and
unlikely changes are listed in Section \ref{SecUchange}.

\subsection{Anticipated Changes} \label{SecAchange}

Anticipated changes are the source of the information that is to be hidden
inside the modules. Ideally, changing one of the anticipated changes will only
require changing the one module that hides the associated decision. The approach
adapted here is called design for
change.

\begin{description}
  \item[\refstepcounter{acnum} \actheacnum \label{ac1}:] The metrics used to assess a domain's state of practice may be updated (e.g. Addition of new metrics, such as code coverage percentage)
  \item[\refstepcounter{acnum} \actheacnum \label{ac2}:] The system should support extension to user authentication (e.g. Using two-factor authentication on top of username and password)
  \item[\refstepcounter{acnum} \actheacnum \label{ac3}:] The ranking algorithm used within a domain for packages may be changed (e.g. Alternatives to AHP, such as BWM, SSB, etc)
  \item[\refstepcounter{acnum} \actheacnum \label{ac4}:] The comparison algorithms used between domains may be changed
  \item[\refstepcounter{acnum} \actheacnum \label{ac5}:] The user access roles available might be expanded (e.g. Admin, User, Contributor)
  \item[\refstepcounter{acnum} \actheacnum \label{ac6}:] Additional languages might be supported (e.g. French, Simplified Chinese)
  \item[\refstepcounter{acnum} \actheacnum \label{ac7}:] The APIs used to extract repository metrics
  \item[\refstepcounter{acnum} \actheacnum \label{ac8}:] The visualization libraries used to display and export data
  \item[\refstepcounter{acnum} \actheacnum \label{ac9}:] Additional export and import file formatting may be added (e.g. CSV, Excel)
\end{description}
\subsection{Unlikely Changes} \label{SecUchange}

The module design should be as general as possible. However, a general system is
more complex. Sometimes this complexity is not necessary. Fixing some design
decisions at the system architecture stage can simplify the software design. If
these decision should later need to be changed, then many parts of the design
will potentially need to be modified. Hence, it is not intended that these
decisions will be changed.

\begin{description}
  \item[\refstepcounter{ucnum} \uctheucnum \label{uc1}:] The schema structure of our domain and metrics systems are designed for long-term use and allows for the addition of new metric types
  \item[\refstepcounter{ucnum} \uctheucnum \label{uc2}:] The platform of the tool will remain as a web application
  \item[\refstepcounter{ucnum} \uctheucnum \label{uc3}:] The development stack of our web application will remain the same (\href{https://react.dev/}{React}, \href{https://www.djangoproject.com/}{Django})
  \item[\refstepcounter{ucnum} \uctheucnum \label{uc4}:] Input types of existing metrics, as outlined in the \href{https://arxiv.org/pdf/2110.11575}{Methodology paper}
  \item[\refstepcounter{ucnum} \uctheucnum \label{uc5}:] Popular browser compatibility is expected to remain unchanged (e.g. Chrome, Firefox, Edge)
  \item[\refstepcounter{ucnum} \uctheucnum \label{uc6}:] Tool owner, Dr. Spencer Smith, is not expected to change for the duration of the tool's lifespan
\end{description}

\section{Module Hierarchy} \label{SecMH}

This section provides an overview of the module design. Modules are summarized
in a hierarchy decomposed by secrets in Table \ref{TblMH}. The modules listed
below, which are leaves in the hierarchy tree, are the modules that will
actually be implemented.

\begin{description}
\item [\refstepcounter{mnum} \mthemnum \label{mHH1}:] Hardware Hiding Module
\item [\refstepcounter{mnum} \mthemnum \label{mHH2}:] Browser Module
\item [\refstepcounter{mnum} \mthemnum \label{mBH1}:] Domain View Module
\item [\refstepcounter{mnum} \mthemnum \label{mBH2}:] Data Edit Module
\item [\refstepcounter{mnum} \mthemnum \label{mBH3}:] User Authentication Module
\item [\refstepcounter{mnum} \mthemnum \label{mBH4}:] User Role Access Module
\item [\refstepcounter{mnum} \mthemnum \label{mBH5}:] Automated Metrics Module
\item [\refstepcounter{mnum} \mthemnum \label{mBH6}:] Localization Module
\item [\refstepcounter{mnum} \mthemnum \label{mBH7}:] Domain Management Module
\item [\refstepcounter{mnum} \mthemnum \label{mSD1}:] System API Gateway Module
\item [\refstepcounter{mnum} \mthemnum \label{mSD2}:] Ranking Algorithm Module
\item [\refstepcounter{mnum} \mthemnum \label{mSD3}:] Graphing Module
\item [\refstepcounter{mnum} \mthemnum \label{mSD4}:] File Import Module
\item [\refstepcounter{mnum} \mthemnum \label{mSD5}:] File Export Module
\item [\refstepcounter{mnum} \mthemnum \label{mSD6}:] Repository API Module
\item [\refstepcounter{mnum} \mthemnum \label{mSD7}:] Domain Comparison Module
\item [\refstepcounter{mnum} \mthemnum \label{mSD8}:] Database Persistence Module

\end{description}


\begin{table}[h!]
\centering
\begin{tabular}{p{0.3\textwidth} p{0.6\textwidth}}
\toprule
\textbf{Level 1} & \textbf{Level 2}\\
\midrule

{Hardware-Hiding Module} & Browser Module\\
\midrule
\multirow{7}{0.3\textwidth}{Behaviour-Hiding Module} &  Domain View Module \\
& Data Edit Module \\
& User Authentication Module\\
& User Role Access Module\\
& Automated Metrics Module\\
& Localization Module\\
& Domain Management Module \\
\midrule

\multirow{3}{0.3\textwidth}{Software Decision Module} & {System API Gateway Module}\\
& Ranking Algorithm Module\\
& Graphing Module\\
& File Import Module\\
& File Export Module\\
& Repository API Module\\
& Domain Comparison Module\\
& Database Persistence Module \\
\bottomrule

\end{tabular}
\caption{Module Hierarchy}
\label{TblMH}
\end{table}

\section{Connection Between Requirements and Design} \label{SecConnection}

The design of the system is intended to satisfy the requirements developed in
the SRS. In this stage, the system is decomposed into modules. The connection
between requirements and modules is listed in Table~\ref{TblRT}.

\section{Module Decomposition} \label{SecMD}

Modules are decomposed according to the principle of ``information hiding''
proposed by \citet{ParnasEtAl1984}. The \emph{Secrets} field in a module
decomposition is a brief statement of the design decision hidden by the
module. The \emph{Services} field specifies \emph{what} the module will do
without documenting \emph{how} to do it. For each module, a suggestion for the
implementing software is given under the \emph{Implemented By} title. If the
entry is \emph{OS}, this means that the module is provided by the operating
system or by standard programming language libraries.  \emph{\progname{}} means the
module will be implemented by the \progname{} software.

Only the leaf modules in the hierarchy have to be implemented. If a dash
(\emph{--}) is shown, this means that the module is not a leaf and will not have
to be implemented.

\subsection{Hardware Hiding Modules (\mref{mHH1})}
\begin{description}
\item[Secrets:] The data structure and algorithm used to implement the virtual
  hardware.
\item[Services:] Serves as a virtual hardware used by the rest of the
  system. This module provides the interface between the hardware and the
  software. So, the system can use it to display outputs or to accept inputs.
\item[Implemented By:] OS
\end{description}

\subsubsection{Browser Module (\mref{mHH2})}
\begin{description}
\item[Secrets:] The data structure and algorithm used to implement the browser, which is outside of the scope of the project.
\item[Services:] The browser allows all users of the product to view and retrieve the project through the internet, and displays the contents for use.
\item[Implemented By:] Web browser
\end{description}

\subsection{Behaviour-Hiding Module}

\begin{description}
\item[Secrets:]The contents of the required behaviours.
\item[Services:]Includes programs that provide externally visible behaviour of
  the system as specified in the software requirements specification (\href{https://github.com/thaafei/DomainX/blob/main/docs/SRS/SRS.pdf}{SRS})
  documents. This module serves as a communication layer between the
  hardware-hiding module and the software decision module. The programs in this
  module will need to change if there are changes in the SRS.
\item[Implemented By:] --
\end{description}

\subsubsection{ Domain View Module (\mref{mBH1})}
\begin{description}
\item[Secrets:] The data structures and algorithms to retrieve and display domain metrics and packages.
\item[Services:] Displays domain content, including it's corresponding packages and metrics. 
\item[Implemented By:] \projname
\item[Type of Module:] ADT
\end{description}

\subsubsection{ Data Edit Module (\mref{mBH2})}
\begin{description}
\item[Secrets:] The data structures and algorithms to edit and update metrics of a domain. 
\item[Services:]  Handles and validates user inputs of domain/package/metric data. Provides user feedback on errors (invalid type, missing field, etc)
\item[Implemented By:] \projname
\item[Type of Module:] ADT
\end{description}

\subsubsection{ User Authentication Module (\mref{mBH3})}
\begin{description}
\item[Secrets:] The data structures and algorithms used to securely store, validate and manage user credentials.
\item[Services:] Provides user registration, login, and session management services.
\item[Implemented By:]  \projname
\item[Type of Module:] Library
\end{description}

\subsubsection{ User Role Access Module (\mref{mBH4})}
\begin{description}
\item[Secrets:] The data structures and algorithms used to store and validate users access level within the system.
\item[Services:] Provides user role and capabilities related to the role.
\item[Implemented By:]  \projname
\item[Type of Module:] ADT
\end{description}

\subsubsection{ Automated Metrics Module (\mref{mBH5})}
\begin{description}
\item[Secrets:] The data structures and algorithms used for retrieving automatable metrics.
\item[Services:] Handles any automated data entry into system, uses Repository API module and File Import Module to automate data input.
\item[Implemented By:]  \projname
\item[Type of Module:] Library
\end{description}

\subsubsection{Localization Module (\mref{mBH6})}
\begin{description}
\item[Secrets:] The language files, translation mappings, and locale formatting mechanisms.
\item[Services:] Provides translation and formatting of UI text and date/number formats.
\item[Implemented By:] \href{https://react.i18next.com/}{react-i18next}
\item[Type of Module:] Library
\end{description}

\subsubsection{ Domain Management Module (\mref{mBH7})}
\begin{description}
\item[Secrets:] The data structures and algorithms that handles the creation and publication of domains.
\item[Services:] Provides versioning and publication of completed domains. 
\item[Implemented By:] \projname
\item[Type of Module:] ADT
\end{description}

\subsection{Software Decision Module}
\begin{description}
\item[Secrets:] The design decision based on mathematical theorems, physical
  facts, or programming considerations. The secrets of this module are
  \emph{not} described in the SRS.
\item[Services:] Includes data structure and algorithms used in the system that
  do not provide direct interaction with the user. 
  % Changes in these modules are more likely to be motivated by a desire to
  % improve performance than by externally imposed changes.
\item[Implemented By:] --
\end{description}

\subsubsection{System API Gateway Module (\mref{mSD1})}
\begin{description}
\item[Secrets:] The business logic that coordinates the flow of communication between modules, and the frontend, backend, and database.
\item[Services:] Manages application state and serves as the central communication hub for the system
\item[Implemented By:] \href{https://react.dev/}{React}, \href{https://www.djangoproject.com/}{Django}
\item[Type of Module:] Abstract Object
\end{description}


\subsubsection{Ranking Algorithm Module (\mref{mSD2})}
\begin{description}
\item[Secrets:] Ranking algorithms and weights (AHP, BWM, SSB).
\item[Services:] Computes comparative rankings using configurable methods.
\item[Implemented By:] \href{https://pypi.org/project/AHPy/}{AHPy}, \projname
\item[Type of Module:] Abstract Object
\end{description}

\subsubsection{Graphing Module (\mref{mSD3})}
\begin{description}
\item[Secrets:] The data structures and algorithms used for graphing data.
\item[Services:] Provides the functions related to graphing.
\item[Implemented By:] \href{https://pypi.org/project/matplotlib/}{Matplotlib}
\item[Type of Module:] Abstract Object
\end{description}

\subsubsection{File Import Module (\mref{mSD4})}
\begin{description}
\item[Secrets:] The data structures and algorithms used to import data of varying formats (e.g. Excel, CSV).
\item[Services:] Facilitates the process of processing a inputted file into data for use by the rest of the system.
\item[Implemented By:] \href{https://pypi.org/project/pandas/}{pandas}
\item[Type of Module:] Abstract Object
\end{description}

\subsubsection{File Export Module (\mref{mSD5})}
\begin{description}
\item[Secrets:] The data structures and algorithms used to export data in varying formats (e.g. Excel, CSV).
\item[Services:] Facilitates the process of transforming system data into an exportable format (e.g. Excel, CSV).
\item[Implemented By:] \href{https://pypi.org/project/pandas/}{pandas}
\item[Type of Module:] Abstract Object
\end{description}

\subsubsection{Repository API Module (\mref{mSD6})}
\begin{description}
\item[Secrets:] API endpoints, tokens, and rate limit strategies.
\item[Services:] Fetches metrics and metadata from external repositories (e.g., GitHub, GitLab).
\item[Implemented By:] \projname
\item[Type of Module:] Abstract Object
\end{description}

\subsubsection{Domain Comparison Module (\mref{mSD7})}
\begin{description}
\item[Secrets:] The algorithms and parameters used for comparing domains.
\item[Services:] Computes and visualizes comparisons across multiple domains.
\item[Implemented By:] \projname
\item[Type of Module:] Abstract Object
\end{description}

\subsubsection{Database Persistence Module (\mref{mSD8})}
\begin{description}
\item[Secrets:] The algorithms and parameters used for interacting with the database.
\item[Services:] Provides database methods related to updating and querying the stored data. As well as the connection to the database itself.
\item[Implemented By:] \href{https://www.mysql.com/}{MySQL}
\item[Type of Module:] Abstract Object
\end{description}

This section shows two traceability matrices: between the modules and the
requirements and between the modules and the anticipated changes.

% the table should use mref, the requirements should be named, use something
% like fref
\begin{table}[H]
\centering
\begin{tabular}{p{0.2\textwidth} p{0.6\textwidth}}
\toprule
\textbf{Req.} & \textbf{Modules}\\
\midrule
FR1 & \mref{mBH4}\\
FR2 & \mref{mBH3}, \mref{mBH4}\\
FR3 & \mref{mBH3}\\
FR4 & \mref{mBH3}\\
FR5 & \mref{mBH2}, \mref{mSD8}  \\
FR6 & \mref{mBH7}  \\
FR7 & \mref{mBH2}, \mref{mBH4}, \mref{mSD8} \\
FR8 & \mref{mBH2} \\
FR9 & \mref{mBH2}, \mref{mBH4}, \mref{mBH5}, \mref{mSD4} \\
FR10 & \mref{mBH5}, \mref{mSD4}, \mref{mSD6} \\
FR11 & \mref{mBH1}, \mref{mSD8}\\
FR12 & \mref{mSD2}, \mref{mBH1}\\
FR13 & \mref{mSD3}\\
FR14 & \mref{mSD5}\\
FR15 & \mref{mSD3}, \mref{mSD5}\\
\bottomrule
\end{tabular}
\caption{Trace Between Functional Requirements and Modules}
\label{TblRT}
\end{table}

\begin{table}[H]
\centering
\begin{tabular}{p{0.2\textwidth} p{0.6\textwidth}}
\toprule
\textbf{Req.} & \textbf{Modules}\\
\midrule
LF-AR1 & \mref{mHH}, \mref{mInput}, \mref{mParams}, \mref{mControl}\\
LF-AR2 & \mref{mInput}, \mref{mParams}\\
LF-AR3 & \mref{mVerify}\\
LF-AR4 & \mref{mOutput}, \mref{mControl}\\
LF-AR5 & \mref{mOutput}, \mref{mODEs}, \mref{mControl}, \mref{mSeqDS}, \mref{mSolver}, \mref{mPlot}\\
LF-AR6 & \mref{mOutput}, \mref{mODEs}, \mref{mControl}, \mref{mSeqDS}, \mref{mSolver}, \mref{mPlot}\\
LF-SR1 & \mref{mOutput}, \mref{mEnergy}, \mref{mControl}, \mref{mSeqDS}, \mref{mPlot}\\
LF-SR2 & \mref{mOutput}, \mref{mEnergy}, \mref{mControl}, \mref{mSeqDS}, \mref{mPlot}\\
LF-SR3 & \mref{mVerifyOut}\\
UH-EU1  & \mref{mOutput}, \mref{mODEs}, \mref{mControl}\\
UH-EU2  & \mref{mOutput}, \mref{mODEs}, \mref{mEnergy}, \mref{mControl}\\
UH-EU3  & \mref{mOutput}, \mref{mODEs}, \mref{mEnergy}, \mref{mControl}\\
UH-LR1  & \mref{mOutput}, \mref{mODEs}, \mref{mEnergy}, \mref{mControl}\\
UH-LR2  & \mref{mOutput}, \mref{mODEs}, \mref{mEnergy}, \mref{mControl}\\
UH-UP1  & \mref{mOutput}, \mref{mODEs}, \mref{mEnergy}, \mref{mControl}\\
UH-UP2 & \mref{mOutput}, \mref{mODEs}, \mref{mEnergy}, \mref{mControl}\\
UH-UP3 & \mref{mOutput}, \mref{mODEs}, \mref{mEnergy}, \mref{mControl}\\
UH-AR1 & \mref{mOutput}, \mref{mODEs}, \mref{mEnergy}, \mref{mControl}\\
UH-AR2 & \mref{mOutput}, \mref{mODEs}, \mref{mEnergy}, \mref{mControl}\\
PR-SL1 & \mref{mOutput}, \mref{mODEs}, \mref{mEnergy}, \mref{mControl}\\
PR-SC1 & \mref{mOutput}, \mref{mODEs}, \mref{mEnergy}, \mref{mControl}\\
PR-SC2 & \mref{mOutput}, \mref{mODEs}, \mref{mEnergy}, \mref{mControl}\\
PR-SC3 & \mref{mOutput}, \mref{mODEs}, \mref{mEnergy}, \mref{mControl}\\
PR-PA1 & \mref{mOutput}, \mref{mODEs}, \mref{mEnergy}, \mref{mControl}\\
PR-RFT1 & \mref{mOutput}, \mref{mODEs}, \mref{mEnergy}, \mref{mControl}\\
PR-RFT2 & \mref{mOutput}, \mref{mODEs}, \mref{mEnergy}, \mref{mControl}\\
PR-RFT3 & \mref{mOutput}, \mref{mODEs}, \mref{mEnergy}, \mref{mControl}\\
PR-CR1 & \mref{mOutput}, \mref{mODEs}, \mref{mEnergy}, \mref{mControl}\\
PR-CR2 & \mref{mOutput}, \mref{mODEs}, \mref{mEnergy}, \mref{mControl}\\
PR-CR3 & \mref{mOutput}, \mref{mODEs}, \mref{mEnergy}, \mref{mControl}\\
PR-SE1 & \mref{mOutput}, \mref{mODEs}, \mref{mEnergy}, \mref{mControl}\\
PR-SE2 & \mref{mOutput}, \mref{mODEs}, \mref{mEnergy}, \mref{mControl}\\
PR-LR1 & \mref{mOutput}, \mref{mODEs}, \mref{mEnergy}, \mref{mControl}\\
PR-LR2 & \mref{mOutput}, \mref{mODEs}, \mref{mEnergy}, \mref{mControl}\\
OE-EPE1 & \mref{mOutput}, \mref{mODEs}, \mref{mEnergy}, \mref{mControl}\\
OE-EPE2 & \mref{mOutput}, \mref{mODEs}, \mref{mEnergy}, \mref{mControl}\\
OE-EPE3 & \mref{mOutput}, \mref{mODEs}, \mref{mEnergy}, \mref{mControl}\\
OE-WE1 & \mref{mOutput}, \mref{mODEs}, \mref{mEnergy}, \mref{mControl}\\
OE-WE2 & \mref{mOutput}, \mref{mODEs}, \mref{mEnergy}, \mref{mControl}\\
OE-WE3 & \mref{mOutput}, \mref{mODEs}, \mref{mEnergy}, \mref{mControl}\\
OE-IA1 & \mref{mOutput}, \mref{mODEs}, \mref{mEnergy}, \mref{mControl}\\
OE-IA2 & \mref{mOutput}, \mref{mODEs}, \mref{mEnergy}, \mref{mControl}\\
OE-IA3 & \mref{mOutput}, \mref{mODEs}, \mref{mEnergy}, \mref{mControl}\\
OE-IA4 & \mref{mOutput}, \mref{mODEs}, \mref{mEnergy}, \mref{mControl}\\
SR-AC1 & \mref{mOutput}, \mref{mODEs}, \mref{mEnergy}, \mref{mControl}\\
SR-AC2 & \mref{mOutput}, \mref{mODEs}, \mref{mEnergy}, \mref{mControl}\\
SR-AC3 & \mref{mOutput}, \mref{mODEs}, \mref{mEnergy}, \mref{mControl}\\
SR-AC4 & \mref{mOutput}, \mref{mODEs}, \mref{mEnergy}, \mref{mControl}\\
SR-AC5 & \mref{mOutput}, \mref{mODEs}, \mref{mEnergy}, \mref{mControl}\\
SR-AC6 & \mref{mOutput}, \mref{mODEs}, \mref{mEnergy}, \mref{mControl}\\
SR-INT1 & \mref{mOutput}, \mref{mODEs}, \mref{mEnergy}, \mref{mControl}\\
SR-INT2 & \mref{mOutput}, \mref{mODEs}, \mref{mEnergy}, \mref{mControl}\\
SR-INT3 & \mref{mOutput}, \mref{mODEs}, \mref{mEnergy}, \mref{mControl}\\
SR-INT4 & \mref{mOutput}, \mref{mODEs}, \mref{mEnergy}, \mref{mControl}\\
SR-INT5 & \mref{mOutput}, \mref{mODEs}, \mref{mEnergy}, \mref{mControl}\\
SR-INT6 & \mref{mOutput}, \mref{mODEs}, \mref{mEnergy}, \mref{mControl}\\
SR-INT7 & \mref{mOutput}, \mref{mODEs}, \mref{mEnergy}, \mref{mControl}\\
SR-P1 & \mref{mOutput}, \mref{mODEs}, \mref{mEnergy}, \mref{mControl}\\
SR-P2 & \mref{mOutput}, \mref{mODEs}, \mref{mEnergy}, \mref{mControl}\\
SR-AU1 & \mref{mOutput}, \mref{mODEs}, \mref{mEnergy}, \mref{mControl}\\
SR-AU2 & \mref{mOutput}, \mref{mODEs}, \mref{mEnergy}, \mref{mControl}\\
SR-AU3 & \mref{mOutput}, \mref{mODEs}, \mref{mEnergy}, \mref{mControl}\\
SR-IM1 & \mref{mOutput}, \mref{mODEs}, \mref{mEnergy}, \mref{mControl}\\
SR-IM2 & \mref{mOutput}, \mref{mODEs}, \mref{mEnergy}, \mref{mControl}\\
SR-IM3 & \mref{mOutput}, \mref{mODEs}, \mref{mEnergy}, \mref{mControl}\\
SR-IM4 & \mref{mOutput}, \mref{mODEs}, \mref{mEnergy}, \mref{mControl}\\

\bottomrule
\end{tabular}
\caption{Trace Between Non-Functional Requirements and Modules}
\label{TblRT}
\end{table}



\begin{table}[H]
\centering
\begin{tabular}{p{0.2\textwidth} p{0.6\textwidth}}
\toprule
\textbf{AC} & \textbf{Modules}\\
\midrule
\acref{ac1} & \mref{mBH1}, \mref{mBH2}, \mref{mSD8}\\
\acref{ac2} & \mref{mBH3}\\
\acref{ac3} & \mref{mSD2}\\
\acref{ac4} & \mref{mSD7}\\
\acref{ac5} & \mref{mBH4}\\
\acref{ac6} & \mref{mBH6}\\
\acref{ac7} & \mref{mSD6}\\
\acref{ac8} & \mref{mSD3}\\
\acref{ac9} & \mref{mSD4},  \mref{mSD5}\\
\bottomrule
\end{tabular}
\caption{Trace Between Anticipated Changes and Modules}
\label{TblACT}
\end{table}


\section{Use Hierarchy Between Modules} \label{SecUse}

In this section, the uses hierarchy between modules is
provided. \citet{Parnas1978} said of two programs A and B that A {\em uses} B if
correct execution of B may be necessary for A to complete the task described in
its specification. That is, A {\em uses} B if there exist situations in which
the correct functioning of A depends upon the availability of a correct
implementation of B.  Figure \ref{FigUH} illustrates the use relation between
the modules. It can be seen that the graph is a directed acyclic graph
(DAG). Each level of the hierarchy offers a testable and usable subset of the
system, and modules in the higher level of the hierarchy are essentially simpler
because they use modules from the lower levels.

\wss{The uses relation is not a data flow diagram.  In the code there will often
be an import statement in module A when it directly uses module B.  Module B
provides the services that module A needs.  The code for module A needs to be
able to see these services (hence the import statement).  Since the uses
relation is transitive, there is a use relation without an import, but the
arrows in the diagram typically correspond to the presence of import statement.}

\wss{If module A uses module B, the arrow is directed from A to B.}

\begin{figure}[H]
\centering
%\includegraphics[width=0.7\textwidth]{UsesHierarchy.png}
\caption{Use hierarchy among modules}
\label{FigUH}
\end{figure}

%\section*{References}

\section{User Interfaces}

\wss{Design of user interface for software and hardware.  Attach an appendix if
needed. Drawings, Sketches, Figma}

\section{Design of Communication Protocols}

\wss{If appropriate}

\section{Timeline}

\wss{Schedule of tasks and who is responsible}

\wss{You can point to GitHub if this information is included there}

\bibliographystyle {plainnat}
\bibliography{../../../refs/References}

\newpage{}

\end{document}