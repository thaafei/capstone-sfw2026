\documentclass{article}

\usepackage{booktabs}
\usepackage{tabularx}
\usepackage{hyperref}

\hypersetup{
    colorlinks=true,       % false: boxed links; true: colored links
    linkcolor=red,          % color of internal links (change box color with linkbordercolor)
    citecolor=green,        % color of links to bibliography
    filecolor=magenta,      % color of file links
    urlcolor=cyan           % color of external links
}

\title{Hazard Analysis\\\progname}

\author{\authname}

\date{}

\input{../Comments}
%% Common Parts

\newcommand{\progname}{SFWRENG 4G06 - Capstone Design Process}
\newcommand{\authname}{\textbf{Team 17, DomainX} \\
\\ Awurama Nyarko
\\ Haniye Hamidizadeh
\\ Fei Xie
\\ Ghena Hatoum             
}
\newcommand{\projname}{DomainX}
\usepackage{hyperref}
    \hypersetup{colorlinks=true, linkcolor=blue, citecolor=blue, filecolor=blue,
                urlcolor=blue, unicode=false}
    \urlstyle{same}
                                


\begin{document}

\maketitle
\thispagestyle{empty}

~\newpage

\pagenumbering{roman}

\begin{table}[hp]
\caption{Revision History} \label{TblRevisionHistory}
\begin{tabularx}{\textwidth}{llX}
\toprule
\textbf{Date} & \textbf{Developer(s)} & \textbf{Change}\\
\midrule
Date1 & Name(s) & Description of changes\\
Date2 & Name(s) & Description of changes\\
... & ... & ...\\
\bottomrule
\end{tabularx}
\end{table}

~\newpage

\tableofcontents

~\newpage

\pagenumbering{arabic}

\wss{You are free to modify this template.}

\section{Introduction}

\wss{You can include your definition of what a hazard is here.}

\section{Scope and Purpose of Hazard Analysis}

\wss{You should say what \textbf{loss} could be incurred because of the
hazards.}

\section{System Boundaries and Components}

\wss{Dividing the system into components will help you brainstorm the hazards.
You shouldn't do a full design of the components, just get a feel for the major
ones.  For projects that involve hardware, the components will typically include
each individual piece of hardware.  If your software will have a database, or an
important library, these are also potential components.}

\section{Critical Assumptions}

\wss{These assumptions that are made about the software or system.  You should
minimize the number of assumptions that remove potential hazards.  For instance,
you could assume a part will never fail, but it is generally better to include
this potential failure mode.}

\section{Failure Mode and Effect Analysis}

\wss{Include your FMEA table here. This is the most important part of this document.}
\wss{The safety requirements in the table do not have to have the prefix SR.
The most important thing is to show traceability to your SRS. You might trace to
requirements you have already written, or you might need to add new
requirements.}
\wss{If no safety requirement can be devised, other mitigation strategies can be
entered in the table, including strategies involving providing additional
documentation, and/or test cases.}

\section{Safety and Security Requirements}
The hazard analysis revealed several safety-related requirements that were not fully covered in the SRS.
These requirements aim to prevent data loss, improve user experience during failures, and keep the tool reliable.

\subsection*{Clear Error Feedback and Account Recovery}
The system must show clear error messages (e.g., wrong password, network failure) and give users a way to reset their credentials if they forget them.

\subsection*{Correct Role Assignment}
User roles (Viewer / Contributor) must be assigned and saved correctly at signup or when updated by an admin so that permissions are always accurate.

\subsection*{Protecting Manual Edits from Automated Updates}
If automated data refreshes could overwrite a user’s manual entry, the system must warn the user or ask for confirmation before replacing the data.

\subsection*{Preventing Conflicting Edits}
When two people try to edit the same record at the same time, the tool must either block one of the saves or clearly warn the users to avoid losing data.

\subsection*{Consistent Visualization}
Publishing visualizations must not happen at the same time as ongoing data edits.
The system should either block publishing during edits or enforce a short downtime so charts and scores always reflect a stable dataset.

\subsection*{Reliable Download of Data and Visuals}
If an export (CSV, PNG, etc.) fails or the file is corrupted, the tool must alert the user and, where possible, let them retry the download.

\section{Roadmap}

\section{Roadmap}

For the Capstone timeline, we will focus on the safety features that are needed to make the tool work properly:

\begin{itemize}
    \item Login with account recovery so that users can sign in securely.
    \item Correct role assignment (Viewer / Contributor) so only authorized users can edit data.
    \item A basic warning when automated updates might overwrite manual edits.
    \item Reliable downloads for tables and charts, with a simple error message if something goes wrong.
\end{itemize}

Some features will be added later as future improvements:

\begin{itemize}
    \item A better system to prevent publishing visualizations while edits are happening (for example, temporary downtime or blocking the publish button).
    \item A stronger solution for conflicting edits, such as proper locking or live conflict warnings.
    \item More advanced error handling for things like failed downloads or export retries.
\end{itemize}


\newpage{}

\section*{Appendix --- Reflection}

\wss{Not required for CAS 741}

\input{../Reflection.tex}

\begin{enumerate}
    \item What went well while writing this deliverable?
    \\\textbf{Haniye:} One thing that went well was that looking at the hazards helped us spot requirements we hadn’t considered before. Thinking through the risks made certain scenarios much clearer, like what happens if login fails or downloaded file is corrupted.
    \item What pain points did you experience during this deliverable, and how
    did you resolve them?
    \\\textbf{Haniye:} One challenge was figuring out which safety features to prioritize for the Capstone demo and which to leave for later. After looking at the project scope, we agreed to focus on the most essential features that would keep the tool stable and reliable for the demo.
    \item Which of your listed risks had your team thought of before this
    deliverable, and which did you think of while doing this deliverable? For
    the latter ones (ones you thought of while doing the Hazard Analysis), how
    did they come about?
    \\\textbf{Haniye:} Some hazards were already on our radar but only partially covered in the original requirements. For example, we had noted the risk of conflicting edits and planned to keep a record of changes, but we hadn’t considered that the system should also block or warn other users editing the same record. Similarly, we knew we’d have different user roles, but we overlooked the need to handle role assignment correctly in the database. Other risks, like login failures or charts showing incomplete data, only came up as we worked through the hazard analysis and FMEA table.
    \item Other than the risk of physical harm (some projects may not have any
    appreciable risks of this form), list at least 2 other types of risk in
    software products. Why are they important to consider?
    \\\textbf{Haniye:} N/A
\end{enumerate}

\end{document}