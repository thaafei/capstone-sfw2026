\documentclass{article}

\usepackage{booktabs}
\usepackage{tabularx}
\usepackage{hyperref}

\hypersetup{
    colorlinks=true,       % false: boxed links; true: colored links
    linkcolor=red,          % color of internal links (change box color with linkbordercolor)
    citecolor=green,        % color of links to bibliography
    filecolor=magenta,      % color of file links
    urlcolor=cyan           % color of external links
}

\title{Hazard Analysis\\\progname}

\author{\authname}

\date{}

\input{../Comments}
%% Common Parts

\newcommand{\progname}{SFWRENG 4G06 - Capstone Design Process}
\newcommand{\authname}{\textbf{Team 17, DomainX} \\
\\ Awurama Nyarko
\\ Haniye Hamidizadeh
\\ Fei Xie
\\ Ghena Hatoum             
}
\newcommand{\projname}{DomainX}
\usepackage{hyperref}
    \hypersetup{colorlinks=true, linkcolor=blue, citecolor=blue, filecolor=blue,
                urlcolor=blue, unicode=false}
    \urlstyle{same}
                                


\begin{document}

\maketitle
\thispagestyle{empty}

~\newpage

\pagenumbering{roman}

\begin{table}[hp]
\caption{Revision History} \label{TblRevisionHistory}
\begin{tabularx}{\textwidth}{llX}
\toprule
\textbf{Date} & \textbf{Developer(s)} & \textbf{Change}\\
\midrule
Oct 4, 2025 & Awurama Nyarko & Introduction, Scope, Critical Assumptions\\
Date2 & Name(s) & Description of changes\\
... & ... & ...\\
\bottomrule
\end{tabularx}
\end{table}

~\newpage

\tableofcontents

~\newpage

\pagenumbering{arabic}

\wss{You are free to modify this template.}

\section{Introduction}


This Hazard Analysis identifies and evaluates potential risks associated with
the Neural Network Libraries (NNL) Assessment Tool, a web-based application that
automates evidence collection, data storage, and visualization for assessing
open-source neural network libraries. 

In this context, a hazard is defined as any condition, event, or design
decision that could lead to loss of data integrity, software malfunction,
degraded performance, or failure to meet stakeholder requirements.

The tool integrates React (frontend), Flask (backend), a relational database
(e.g., MySQL), and public Application Programming Interfaces (APIs), such as the
GitHub API, to support automated data gathering, Analytic Hierarchy Process
(AHP)--based ranking, and visualization of software-quality metrics.

Because the system involves data integration, user interaction, and deployment
on university infrastructure, it faces technical and operational hazards (for
example, integration errors, API limits, or performance bottlenecks). This
document identifies such risks early in the lifecycle to protect software
reliability, data integrity, and user experience.

\section{Scope and Purpose of Hazard Analysis}

The purpose of this hazard analysis is to systematically identify potential
risks that could impact the reliability, usability, and delivery of the NNL
Assessment Tool.

Although the tool is non-safety-critical, losses could still occur through:

\begin{itemize}
    \item Data loss or corruption, affecting research integrity.
    \item System downtime, delaying project milestones or access for the research team.
    \item Inaccurate visualizations or metrics, leading to incorrect conclusions in research outputs.
    \item Security breaches, risking exposure of user accounts or evaluation data.
    \item Integration failures, which could prevent essential automation and data collection.
\end{itemize}

These losses would directly reduce the tool’s credibility, hinder academic
progress, and compromise user trust.

The analysis focuses on identifying, classifying, and mitigating these hazards
early to minimize risks and ensure project success.

\section{System Boundaries and Components}

\wss{Dividing the system into components will help you brainstorm the hazards.
You shouldn't do a full design of the components, just get a feel for the major
ones.  For projects that involve hardware, the components will typically include
each individual piece of hardware.  If your software will have a database, or an
important library, these are also potential components.}

\section{Critical Assumptions}

The following assumptions support hazard identification and mitigation:

\begin{itemize}
    \item \textbf{Access to Public APIs:} It is assumed that GitHub API and other
    data sources will remain stable; however, if access limits or outages occur,
    fallback mechanisms (e.g., cached data, manual upload) will be implemented.
    
    \item \textbf{McMaster Infrastructure Availability:} University servers will
    host the tool; if unavailable, contingency hosting (local or alternative
    cloud) will be explored.
    
    \item \textbf{Stable Development Team:} All members remain active; if a
    member becomes unavailable, roles and documentation ensure continuity.
    
    \item \textbf{Non-Safety-Critical Context:} Hazards relate to data and
    usability, not physical harm, but errors could still cause loss of
    credibility or project delays.
    
    \item \textbf{Defined Scope and Requirements:} Requirements remain stable;
    changes will trigger re-assessment of risks.
    
    \item \textbf{Version Control and Standards:} Git workflow reduces
    integration errors, though merge conflicts remain possible; peer reviews
    mitigate these.
    
    \item \textbf{User Feedback Availability:} Testing feedback will be
    accessible; if delayed, internal testing will substitute temporarily.
\end{itemize}

\section{Failure Mode and Effect Analysis}

\wss{Include your FMEA table here. This is the most important part of this document.}
\wss{The safety requirements in the table do not have to have the prefix SR.
The most important thing is to show traceability to your SRS. You might trace to
requirements you have already written, or you might need to add new
requirements.}
\wss{If no safety requirement can be devised, other mitigation strategies can be
entered in the table, including strategies involving providing additional
documentation, and/or test cases.}

\section{Safety and Security Requirements}

\wss{Newly discovered requirements.  These should also be added to the SRS.  (A
rationale design process how and why to fake it.)}

\section{Roadmap}

\wss{Which safety requirements will be implemented as part of the capstone timeline?
Which requirements will be implemented in the future?}

\newpage{}

\section*{Appendix --- Reflection}

\wss{Not required for CAS 741}

\input{../Reflection.tex}

\begin{enumerate}
    \item What went well while writing this deliverable? 
    \item What pain points did you experience during this deliverable, and how
    did you resolve them?
    \item Which of your listed risks had your team thought of before this
    deliverable, and which did you think of while doing this deliverable? For
    the latter ones (ones you thought of while doing the Hazard Analysis), how
    did they come about?
    \item Other than the risk of physical harm (some projects may not have any
    appreciable risks of this form), list at least 2 other types of risk in
    software products. Why are they important to consider?
\end{enumerate}

\end{document}