\documentclass{article}

\usepackage{booktabs}
\usepackage{tabularx}
\usepackage{hyperref}
\usepackage{comment}
\usepackage{enumerate}
\usepackage{adjustbox}
\usepackage{booktabs}
\usepackage{multirow}
\usepackage{makecell}
\usepackage{geometry}
\usepackage{graphicx}
\usepackage[shortlabels]{enumitem}
\usepackage{float}
\usepackage{array}
\usepackage{pdflscape}
\usepackage{tabularx,ragged2e,booktabs,caption}
\usepackage{longtable}
\usepackage{ulem}
\usepackage{ltablex}

\hypersetup{
    colorlinks=true,       % false: boxed links; true: colored links
    linkcolor=red,          % color of internal links (change box color with linkbordercolor)
    citecolor=green,        % color of links to bibliography
    filecolor=magenta,      % color of file links
    urlcolor=cyan           % color of external links
}

\title{Hazard Analysis\\\progname}

\author{\authname}

\date{}

\input{../Comments}
%% Common Parts

\newcommand{\progname}{SFWRENG 4G06 - Capstone Design Process}
\newcommand{\authname}{\textbf{Team 17, DomainX} \\
\\ Awurama Nyarko
\\ Haniye Hamidizadeh
\\ Fei Xie
\\ Ghena Hatoum             
}
\newcommand{\projname}{DomainX}
\usepackage{hyperref}
    \hypersetup{colorlinks=true, linkcolor=blue, citecolor=blue, filecolor=blue,
                urlcolor=blue, unicode=false}
    \urlstyle{same}
                                


\begin{document}

\maketitle
\thispagestyle{empty}

\pagenumbering{roman}

\begin{table}[hp]
\caption{Revision History} \label{TblRevisionHistory}
\begin{tabularx}{\textwidth}{llX}
\toprule
\textbf{Date} & \textbf{Developer(s)} & \textbf{Change}\\
\midrule
October 6, 2025 & Fei Xie & Added FMEA table\\
Date2 & Name(s) & Description of changes\\
... & ... & ...\\
\bottomrule
\end{tabularx}
\end{table}

~\newpage

\tableofcontents

~\newpage


\wss{You are free to modify this template.}

\section{Introduction}

\wss{You can include your definition of what a hazard is here.}

\section{Scope and Purpose of Hazard Analysis}

\wss{You should say what \textbf{loss} could be incurred because of the
hazards.}

\section{System Boundaries and Components}

\wss{Dividing the system into components will help you brainstorm the hazards.
You shouldn't do a full design of the components, just get a feel for the major
ones.  For projects that involve hardware, the components will typically include
each individual piece of hardware.  If your software will have a database, or an
important library, these are also potential components.}

\section{Critical Assumptions}

\wss{These assumptions that are made about the software or system.  You should
minimize the number of assumptions that remove potential hazards.  For instance,
you could assume a part will never fail, but it is generally better to include
this potential failure mode.}

\section{Failure Mode and Effect Analysis}

\newgeometry{margin=0.5in}
\begin{landscape}
  \begin{longtable}{|p{3cm}|p{3cm}|p{4cm}|p{4cm}|p{3cm}|p{2cm}|p{3cm}|}
  \caption{Failure Mode and Effect Analysis} \label{FMEA}\\
  \hline
   Component & Failure Modes & Effects of Failure & Causes of Failure & Recommended Action & SR & Ref.  \\ 
  \hline
  \endfirsthead
  \multicolumn{7}{r}{Table \thetable\ Continued from previous page}\\ 
  \hline
   Design Function & Failure Modes & Effects of Failure & Causes of Failure & Recommended Action & SR & Ref.  \\ 
  \hline
  \endhead
  \multicolumn{7}{r}{{Continued on next page}}\\
  \endfoot
  \multicolumn{7}{r}{{Concluded}}\\
  \endlastfoot
  \multirow{7}{*}{User account} & 
  \begin{enumerate}[leftmargin=*]
    \item User can't login/signup.
    \item Cannot set correct role for user.
    \item User account information is leaked.
  \end{enumerate} & 
  \begin{enumerate}[leftmargin=*]
    \item User cannot access their work.
    \item Refer to H1-1.
    \item User credentials are exposed, exposing them to cyber attack or data scrapers.
  \end{enumerate} &
  \begin{enumerate}[leftmargin=*]
    \item
    \begin{enumerate}
        \item[a)] Integration with database failure.
        \item[b)] User entered incorrect credentials.
    \end{enumerate}
    \item Refer to H1-1.a.
    \item Weak access controls, lack of encryption, or insecure credentials.
  \end{enumerate} &
  \begin{enumerate}[leftmargin=*]
    \item 
    \begin{enumerate}
        \item[a)] Implement automated daily system integration testing.
        \item[b)] Provide user feedback during user actions.
    \end{enumerate}
    \item Refer to H1-1.a, H1-1.b.
    \item Implement Multi-factor authentication and follow industry best practices for security.
  \end{enumerate} &
  \begin{enumerate}[leftmargin=*]
    \item TODO
    \item TODO
    \item TODO
  \end{enumerate} &
  \begin{enumerate}[leftmargin=*]
    \item H1-1
    \item H1-2
    \item H1-3
  \end{enumerate} \\
  \hline
    Domain Creation & 
  \begin{enumerate}[leftmargin=*]
      \item Cannot create new domains
      \item Cannot edit existing domain
  \end{enumerate} & 
  \begin{enumerate}[leftmargin=*]
    \item
    \begin{enumerate}
        \item[a)] User cannot continue their work on the domain
        \item[b)] User will be delayed when writing their anaylsis
    \end{enumerate}
    \item Refer to H2-1.a, H2-1.b.
  \end{enumerate} &
  \begin{enumerate}[leftmargin=*]
    \item  Refer to H1-1.a
    \begin{enumerate}
        \item[a)] User has incorrect role credentials
    \end{enumerate}
    \item Refer to H1-1.a, H2-1.a.
  \end{enumerate} &
  \begin{enumerate}[leftmargin=*]
       \item Refer to H1-1.a, H1-1.b.
       \item  Refer to H1-1.a, H1-1.b.
  \end{enumerate} &
  \begin{enumerate}[leftmargin=*]
       \item TODO
       \item TODO
  \end{enumerate} &
  \begin{enumerate}[leftmargin=*]
       \item H2-1
       \item H2-2
  \end{enumerate} \\
  \hline
  Adding Data to Domain & 
  \begin{enumerate}[leftmargin=*]
      \item User cannot add new datapoint
      \item User cannot update existing datapoint
      \item Automated process overwriting user data unknowningly
      \item Automated data input failure
  \end{enumerate} & 
  \begin{enumerate}[leftmargin=*]
      \item Audio-based analysis is incomplete or fails, impacting the overall data analysis outcome.
      \item Missed events or actions during analysis.
  \end{enumerate} &
  \begin{enumerate}[leftmargin=*]
       \item File corruption, incorrect file format, or lack of access permissions.
       \item Inadequate training on diverse audio samples or background noise interference.
  \end{enumerate} &
  \begin{enumerate}[leftmargin=*]
       \item Validate audio files before analysis, provide user guidelines for supported formats, and log access errors.
       \item Use noise reduction preprocessing, retrain the model with varied audio data.
  \end{enumerate} &
  \begin{enumerate}[leftmargin=*]
       \item FR-VADA1
       \item FR-VADA3
  \end{enumerate} &
  \begin{enumerate}[leftmargin=*]
       \item HA-AAM1
       \item HA-AAM2
  \end{enumerate} \\
  \hline
  Video Recording & 
  \begin{enumerate}[leftmargin=*]
      \item Video recording is blurry
      \item User is not fully visible in the video
  \end{enumerate} & 
  \begin{enumerate}[leftmargin=*]
      \item The video analysis model may miss critical details, leading to inaccurate analysis.
      \item System may be unable to track bias detection accurately if the user is not fully in the camera frame.
  \end{enumerate} &
  \begin{enumerate}[leftmargin=*]
       \item Poor lighting, low-resolution recording settings, or motion blur, camera lens is dirty or obstructed.
       \item User is unaware of proper framing or positioning guidelines.
  \end{enumerate} &
  \begin{enumerate}[leftmargin=*]
       \item Prompt the user to clean their camera lens and verify camera settings and apply post-processing filters.
       \item Provide a camera testing checkpoint to let the user see if they are in frame before starting the assessment
  \end{enumerate} &
  \begin{enumerate}[leftmargin=*]
       \item FR-SS3, FR-SS4
  \end{enumerate} &
  \begin{enumerate}[leftmargin=*]
       \item HA-VR1, HA-VR2
  \end{enumerate} \\
  \hline
  Audio Recording  & 
  \begin{enumerate}[leftmargin=*]
      \item Audio recording has background noise interference
  \end{enumerate} & 
  \begin{enumerate}[leftmargin=*]
      \item Poor quality audio makes it difficult for the model to detect speech or audio events accurately.
  \end{enumerate} &
  \begin{enumerate}[leftmargin=*]
       \item Faulty recording equipment, interference, or poor recording environment.
  \end{enumerate} &
  \begin{enumerate}[leftmargin=*]
       \item Filter noise using software tools, and provide best practices for recording.
  \end{enumerate} &
  \begin{enumerate}[leftmargin=*]
       \item FR-SS2
  \end{enumerate} &
  \begin{enumerate}[leftmargin=*]
       \item HA-AR1
  \end{enumerate} \\
  \hline
  Backend Server & 
  \begin{enumerate}[leftmargin=*]
      \item Data loss during processing
      \item Server crashes due to user overload
  \end{enumerate} & 
  \begin{enumerate}[leftmargin=*]
      \item Partial or complete loss of data during video/audio processing could result in incomplete analysis
      \item Users may be unable to complete the assessment, or the server crashing could destroy user data.
  \end{enumerate} &
  \begin{enumerate}[leftmargin=*]
       \item Server overload or incorrect handling of data transfer.
       \item High traffic overload, memory leaks, or unhandled exceptions.
  \end{enumerate} &
  \begin{enumerate}[leftmargin=*]
       \item Use robust data storage solutions such as a temporary cache before saving or transaction-based logging.
       \item Monitor server health and enable automatic failover mechanisms.
  \end{enumerate} &
  \begin{enumerate}[leftmargin=*]
       \item FR-DSC1, FR-DSC2
       \item PR-CR1, PR-CR2, PR-CR3
  \end{enumerate} &
  \begin{enumerate}[leftmargin=*]
       \item HA-BS1
       \item HA-BS2
  \end{enumerate} \\
  \hline
  User-Facing Application & 
  \begin{enumerate}[leftmargin=*]
      \item Error in navigation structure/flow
      \item Button components aren't clickable across all devices
  \end{enumerate} & 
  \begin{minipage}[t]{\linewidth}
  \begin{enumerate}[leftmargin=*]
      \item Users cannot move through the application smoothly, leading to frustration and a poor user experience.
      \item Users cannot complete quizzes or interact with the UI, preventing task completion and blocking progression.
  \end{enumerate}
  \end{minipage} &
  \begin{minipage}[t]{\linewidth}
  \begin{enumerate}[leftmargin=*]
       \item Incorrect or missing routing logic, broken links, or unhandled navigation events.
       \item UI components not rendering correctly due to screen size, device-specific compatibility issues, or missing accessibility focus states.
  \end{enumerate}
  \end{minipage} &
  \begin{minipage}[t]{\linewidth}
  \begin{enumerate}[leftmargin=*]
       \item Conduct usability testing and implement fallback paths or error boundaries for broken flows.
       \item Validate responsiveness using tools like Chrome DevTools, test across common screen sizes and OS/browser combinations, and implement accessibility-friendly controls.
  \end{enumerate}
  \end{minipage} &
  \begin{minipage}[t]{\linewidth}
  \begin{enumerate}[leftmargin=*]
       \item LF-AR2, LF-AR5
       \item UH-AR1, LF-AR4, MS-AR1
  \end{enumerate}
  \end{minipage} &
  \begin{minipage}[t]{\linewidth}
  \begin{enumerate}[leftmargin=*]
       \item HA-UI1
       \item HA-UI2
  \end{enumerate}
  \end{minipage} \\
  \hline
  \end{longtable}
\end{landscape}
\restoregeometry


\section{Safety and Security Requirements}

\wss{Newly discovered requirements.  These should also be added to the SRS.  (A
rationale design process how and why to fake it.)}

\section{Roadmap}

\wss{Which safety requirements will be implemented as part of the capstone timeline?
Which requirements will be implemented in the future?}

\newpage{}

\section*{Appendix --- Reflection}

\wss{Not required for CAS 741}

\input{../Reflection.tex}

\begin{enumerate}
    \item What went well while writing this deliverable? 
    \item What pain points did you experience during this deliverable, and how
    did you resolve them?
    \item Which of your listed risks had your team thought of before this
    deliverable, and which did you think of while doing this deliverable? For
    the latter ones (ones you thought of while doing the Hazard Analysis), how
    did they come about?
    \item Other than the risk of physical harm (some projects may not have any
    appreciable risks of this form), list at least 2 other types of risk in
    software products. Why are they important to consider?
\end{enumerate}

\end{document}