\subsection*{Fei}
\begin{enumerate}
  \item \textbf{What went well while writing this deliverable?} \\
  Each team member knew what requirements were, and would work on the sections we decided during the team meeting.
  \item \textbf{What pain points did you experience during this deliverable, and how did
  you resolve them?} \\
  Issues getting the rest of the team to review my work and getting their work up for review. Leaving it hard for me to review their work if it's uploaded so late. This makes it hard to structure PRs, since chances are no one is reviewing my PR even with the team assigned as reviewers, and I can't tell if they seen the comments I left on their PR either. Still wasn't able to completely resolve the issues of reviewing, hopefully a team meeting specifically on the expectations of reviews will clear this up.
  \item \textbf{How many of your requirements were inspired by speaking to your client(s) or their proxies (e.g. your peers, stakeholders, potential users)?} \\
  Due to the unique nature of our project, almost all the requirements were inspired by our client, since we are the clients. After reflecting on what we wanted to see when we use the tool in writing the paper, we were able to easily think of requirements.
  \item \textbf{Which of the courses you have taken, or are currently taking, will help your team to be successful with your capstone project.} \\
  3RA3 has definitely helped, as well as 4HC3, in terms of understanding what a good layout would be (Users are drawn to colours in the middle of their vision, but more sensitive to motion on the peripherals)
  \item \textbf{What knowledge and skills will the team collectively need to acquire to 
  successfully complete this capstone project?  Examples of possible knowledge
  to acquire include domain specific knowledge from the domain of your
  application, or software engineering knowledge, mechatronics knowledge or
  computer science knowledge.  Skills may be related to technology, or writing,
  or presentation, or team management, etc.  You should look to identify at
  least one item for each team member.} \\
  I think the team needs to acquire system design knowledge. Although we've all had experience working adding/updating features of pre-existing projects, starting a completely new one and having to think of all the design requirements will be hard. Additionally, due to the research section, we'll also need to get research, and by expansion, writing knowledge.
  \item \textbf{For each of the knowledge areas and skills identified in the previous
  question, what are at least two approaches to acquiring the knowledge or
  mastering the skill?  Of the identified approaches, which will each team
  member pursue, and why did they make this choice?} \\
  For my, to pursue the system design knowledge, I believe the best way is to just do it. Early consideration of the design of the project, as well as determining the systems that'll need integrating is crucial. Luckily, we've already started doing that so the rest will just be trial and error. Also reviewing 3RA3 content when needed.
  For the research/writing knowledge, I'll be reviewing the methodology research paper as well as the existing papers on other domains. As well as consulting with our supervisor during the writing process.
\end{enumerate}
\subsection*{Fatemeh}
\begin{enumerate}
  \item What went well while writing this deliverable? 
  \item What pain points did you experience during this deliverable, and how did
  you resolve them?
  \item How many of your requirements were inspired by speaking to your
  client(s) or their proxies (e.g. your peers, stakeholders, potential users)?
  \item Which of the courses you have taken, or are currently taking, will help
  your team to be successful with your capstone project.
  \item What knowledge and skills will the team collectively need to acquire to
  successfully complete this capstone project?  Examples of possible knowledge
  to acquire include domain specific knowledge from the domain of your
  application, or software engineering knowledge, mechatronics knowledge or
  computer science knowledge.  Skills may be related to technology, or writing,
  or presentation, or team management, etc.  You should look to identify at
  least one item for each team member.
  \item For each of the knowledge areas and skills identified in the previous
  question, what are at least two approaches to acquiring the knowledge or
  mastering the skill?  Of the identified approaches, which will each team
  member pursue, and why did they make this choice?
\end{enumerate}
\subsection*{Ghena}

\begin{enumerate}
  \item What went well while writing this deliverable? 
  \item What pain points did you experience during this deliverable, and how did
  you resolve them?
  \item How many of your requirements were inspired by speaking to your
  client(s) or their proxies (e.g. your peers, stakeholders, potential users)?
  \item Which of the courses you have taken, or are currently taking, will help
  your team to be successful with your capstone project.
  \item What knowledge and skills will the team collectively need to acquire to
  successfully complete this capstone project?  Examples of possible knowledge
  to acquire include domain specific knowledge from the domain of your
  application, or software engineering knowledge, mechatronics knowledge or
  computer science knowledge.  Skills may be related to technology, or writing,
  or presentation, or team management, etc.  You should look to identify at
  least one item for each team member.
  \item For each of the knowledge areas and skills identified in the previous
  question, what are at least two approaches to acquiring the knowledge or
  mastering the skill?  Of the identified approaches, which will each team
  member pursue, and why did they make this choice?
\end{enumerate}
\subsection*{Awurama}

\begin{enumerate}
  \item What went well while writing this deliverable? 
  \item What pain points did you experience during this deliverable, and how did
  you resolve them?
  \item How many of your requirements were inspired by speaking to your
  client(s) or their proxies (e.g. your peers, stakeholders, potential users)?
  \item Which of the courses you have taken, or are currently taking, will help
  your team to be successful with your capstone project.
  \item What knowledge and skills will the team collectively need to acquire to
  successfully complete this capstone project?  Examples of possible knowledge
  to acquire include domain specific knowledge from the domain of your
  application, or software engineering knowledge, mechatronics knowledge or
  computer science knowledge.  Skills may be related to technology, or writing,
  or presentation, or team management, etc.  You should look to identify at
  least one item for each team member.
  \item For each of the knowledge areas and skills identified in the previous
  question, what are at least two approaches to acquiring the knowledge or
  mastering the skill?  Of the identified approaches, which will each team
  member pursue, and why did they make this choice?
\end{enumerate}
