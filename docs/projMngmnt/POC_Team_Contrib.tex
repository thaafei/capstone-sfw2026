\documentclass{article}

\usepackage{float}
\restylefloat{table}

\usepackage{booktabs}

\title{Team Contributions: POC\\\progname}

\author{\authname}

\date{}

\input{../Comments}
%% Common Parts

\newcommand{\progname}{SFWRENG 4G06 - Capstone Design Process}
\newcommand{\authname}{\textbf{Team 17, DomainX} \\
\\ Awurama Nyarko
\\ Haniye Hamidizadeh
\\ Fei Xie
\\ Ghena Hatoum             
}
\newcommand{\projname}{DomainX}
\usepackage{hyperref}
    \hypersetup{colorlinks=true, linkcolor=blue, citecolor=blue, filecolor=blue,
                urlcolor=blue, unicode=false}
    \urlstyle{same}
                                


\begin{document}

\maketitle

This document summarizes the contributions of each team member up to the POC
Demo.  The time period of interest is the time between the beginning of the term
and the POC demo.

\section{Demo Plans}
For our demo, we will walk through the main user flow to show that the core system components are connected and functioning. Starting from the Home page, the user navigates to the Sign-In page and logs in. After signing in, the user is taken to the Visualization page, where they can enter a GitHub repository link. Once submitted, our backend retrieves repository information from the GitHub API, and we display a sample visualization (for example, metrics such as the number of stars, contributors, or commit activity) to demonstrate that data is successfully fetched, processed, and rendered. The demo highlights our core PoC objectives: smooth frontend navigation, backend API integration, and visual representation of retrieved data.



\section{Team Meeting Attendance}

% \wss{For each team member how many team meetings have they attended over the
% time period of interest.  This number should be determined from the meeting
% issues in the team's repo.  The first entry in the table should be the total
% number of team meetings held by the team.}

\begin{table}[H]
\centering
\begin{tabular}{ll}
\toprule
\textbf{Student} & \textbf{Meetings}\\
\midrule
Total & 7\\
Awurama & 5\\
Fei & 6\\
Ghena & 7\\
Haniye & 7\\
\bottomrule
\end{tabular}
\end{table}

There was one extra team meeting between Ghena and Haniye to discuss the POC, so they will both have an extra meeting count.
\section{Supervisor/Stakeholder Meeting Attendance}

% \wss{For each team member how many supervisor/stakeholder team meetings have
% they attended over the time period of interest.  This number should be determined
% from the supervisor meeting issues in the team's repo.  The first entry in the
% table should be the total number of supervisor and team meetings held by the
% team.  If there is no supervisor, there will usually be meetings with
% stakeholders (potential users) that can serve a similar purpose.}

\noindent \textbf{Supervisor's Name: Dr. Spencer Smith}

\begin{table}[H]
\centering
\begin{tabular}{ll}
\toprule
\textbf{Student} & \textbf{Meetings}\\
\midrule
Total & 7\\
Awurama & 4\\
Fei & 6\\
Ghena & 6\\
Haniye & 5\\
\bottomrule
\end{tabular}
\end{table}

For one supervisor meeting, only one member (Ghena) was expected to attend.
One domain expert meeting was called, only one member (Fei) was expected to attend.
\section{Lecture Attendance}

% \wss{For each team member how many lectures have they attended over the time
% period of interest.  This number should be determined from the lecture issues in
% the team's repo. You can find the number of lectures in the time period of
% interest by looking at the
% \href{https://calendar.google.com/calendar/u/0/embed?src=rnboqiaki1k2la7rpu3bn0um58@group.calendar.google.com&ctz=America/Toronto}
% {Google calendar} for the capstone course.}

% \wss{NOTE: There will be approximately 13 lectures between the start of class
% and the POC demos}

\begin{table}[H]
\centering
\begin{tabular}{ll}
\toprule
\textbf{Student} & \textbf{Lectures}\\
\midrule
Total & 8\\
Awurama & 4\\
Fei & 7\\
Ghena & 7\\
Haniye & 4\\
\bottomrule
\end{tabular}
\end{table}

\section{TA Document Discussion Attendance}

% \wss{For each team member how many of the informal document discussion meetings
% with the TA were attended over the time period of interest.}

\noindent \textbf{TA's Name: Tanya Djavaherpour}

\begin{table}[H]
\centering
\begin{tabular}{ll}
\toprule
\textbf{Student} & \textbf{Lectures}\\
\midrule
Total & 3\\
Awurama & 3\\
Fei & 3\\
Ghena & 3\\
Haniye & 3\\
\bottomrule
\end{tabular}
\end{table}

\section{Commits}

% \wss{For each team member how many commits to the main branch have been made
% over the time period of interest.  The total is the total number of commits for
% the entire team since the beginning of the term.  The percentage is the
% percentage of the total commits made by each team member.}

\begin{table}[H]
\centering
\begin{tabular}{lll}
\toprule
\textbf{Student} & \textbf{Commits} & \textbf{Percent}\\
\midrule
Total & 147 & 100\% \\
Awurama & 40& 27\% \\
Fei & 66& 45\% \\
Ghena & 22& 15\% \\
Haniye & 19& 13\% \\
\bottomrule
\end{tabular}
\end{table}

Current commits were captured on October 31st, 2025 for the main branch. This excludes the POC branch that was currently in active development.
\section{Issue Tracker}

% \wss{For each team member how many issues have they authored (including open and
% closed issues (O+C)) and how many have they been assigned (only counting closed
% issues (C only)) over the time period of interest.}

\begin{table}[H]
\centering
\begin{tabular}{lll}
\toprule
\textbf{Student} & \textbf{Authored (O+C)} & \textbf{Assigned (C only)}\\
\midrule
Awurama & 3 & 3 \\
Fei & 15 & 13 \\
Ghena & 2 & 2 \\
Haniye & 2 & 2 \\
\bottomrule
\end{tabular}
\end{table}

Excluding issues related to meetings.
\section{CICD}

We use GitHub Actions to automate the building and compiling the changed latex files in our main branch.
Members are allowed to merge their changes into the feature branch once all comments were resolved.
When at least two members of the team has approved the changes to merge the feature branch into the main branch, members can merge the feature branch into main.\\
We also use GitHub Actions to support the software development workflow. All active development takes place in our release branch, where changes are committed and pushed regularly. When changes are pushed or a pull request is opened to update the release branch or merge it into the main branch, GitHub Actions automatically runs build checks for both the backend (Django) and the frontend (React). These checks ensure that dependencies install correctly and that the application builds without errors. Once the release branch is stable, and at least one team member has reviewed and approved the changes, it is merged into the main branch.


\section{Team Charter Trigger Items}

% \wss{Provide a summary of the quantified triggers identified in the team's
% charter.}
\subsection{Summary of Triggers}
\begin{enumerate}
\item{Attending every weekly Monday virtual meeting during lecture time when there are no lectures. Unless the whole team decides to cancel}
\item{Team members expected to attend all meetings on time, fully prepared and having reviewed relevant materials and the agenda for the meeting}
\item{All deliverables must meet the team's agreed-upon standards}
\item{Deliverables must be reviewed by all team members before submission}
\item{Feedback on deliverables and revisions must be made within 7 days}
\end{enumerate}

% \wss{Provide a list of any violations of the triggers.  If the team wishes, the
% violations can be summarized on aggregate, instead of naming specific team
% members.}
\subsection{Trigger Violations}
\begin{table}[H]
\centering
\begin{tabular}{p{3cm} p{10cm}}
\toprule
\textbf{Trigger} & \textbf{Comment}\\
\midrule
Lack of feedback & Only a few members consistently review and provide feedback on pull requests when requested.\\
Feedback resolution & Some feedback on pull requests are not reviewed or addressed, within the 7-days time frame.\\
Deliverables not reviewed & Deliverables are not reviewed by all team members in a visible way.\\
\bottomrule
\end{tabular}
\end{table}

\subsection{Next Step Plans}
% \wss{Provide a plan to address the violations.  This could include revising the
% triggers, if they are found to be too weak, strong or ambiguous.}
Triggers 4 and 5 must be revised. \\
\textbf{4.} Deliverables must be reviewed by all assigned team members through the "reviewed" confirmation on GitHub, before deliverables are ready to be merged. \\
\textbf{5.} Feedback on deliverables must be given in 24 hours after a review is requested. 
\\\\
Additional trigger will be added:\\
\textbf{6.} All feedback must be resolved at least 2 days before the deadline of the deliverable.\\
If feedback is not provided in time, or resolved in time, a callout will be posted in the team's Discord channel. This will be used as evidence if further issues occur.
\section{Additional Productivity Metrics}
No additional productivity metric was used by the team.

\end{document}